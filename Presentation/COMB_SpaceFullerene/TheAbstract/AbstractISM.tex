\documentclass[12pt]{article}
\usepackage{amsfonts, amsmath, latexsym, vmargin, epsfig}
\usepackage{epsf}
\usepackage{url}
\usepackage{color}
\title{The enumeration of space fullerenes}
\setpapersize{custom}{21cm}{29.7cm}
\setmarginsrb{1.7cm}{1cm}{1.7cm}{3.5cm}{0pt}{0pt}{0pt}{0pt}
%marge gauche, marge haut, marge droite, marge bas.

\author{Mathieu DUTOUR}



\begin{document}
\newcommand{\R}{\ensuremath{\mathbb{R}}}
\newcommand{\N}{\ensuremath{\mathbb{N}}}
\newcommand{\Q}{\ensuremath{\mathbb{Q}}}
\newcommand{\C}{\ensuremath{\mathbb{C}}}
\newcommand{\Z}{\ensuremath{\mathbb{Z}}}
\newcommand{\T}{\ensuremath{\mathbb{T}}}
\newtheorem{proposition}{Proposition}
\newtheorem{theorem}{Theorem}
\newtheorem{corollary}{Corollary}
\newtheorem{lemma}{Lemma}
\newtheorem{problem}{Problem}
\newtheorem{conjecture}{Conjecture}
\newtheorem{claim}{Claim}
\newtheorem{remark}{Remark}
\newtheorem{definition}{Definition}
\newcommand{\qed}{\hfill $\Box$ }
\newcommand{\proof}{\noindent{\bf Proof.}\ \ }

\maketitle


\begin{abstract}
A fullerene is a $3$-valent plane graphs whose faces are pentagons and
hexagons.
A fullerene is said to be IHR if its hexagons are adjacent, i.e. are
adjacent only to pentagons. A {\em space fullerene} is a $4$-valent
tiling of the space by IHR fullerenes. Space fullerenes are interesting
in crystallography and in Discrete geometry.


See below the $4$ IHR fullerene and a space fullerene:
\begin{center}
\begin{minipage}{5.5cm}
\begin{minipage}{25mm}
\centering
\epsfxsize=16mm
\epsfbox{PictureAppli/F1.ps}\par
\textcolor{white}{Bonjour}
\end{minipage}
\hfill\begin{minipage}{25mm}
\centering
\epsfxsize=16mm
\epsfbox{PictureAppli/F2sec.eps}\par
\textcolor{white}{Bonjour}
\end{minipage}
\begin{minipage}{25mm}
\centering
\epsfxsize=16mm
\epsfbox{PictureAppli/F3sec.eps}\par
\textcolor{white}{Bonjour}
\end{minipage}
\hfill\begin{minipage}{25mm}
\centering
\epsfxsize=16mm
\epsfbox{PictureAppli/F4sec.eps}\par
\textcolor{white}{Bonjour}
\end{minipage}
\end{minipage}
\begin{minipage}{6.5cm}
\centering
\resizebox{4cm}{!}{\includegraphics[bb=176 30 420 267, clip]{PictureAppli/fig14.eps}}\par
\end{minipage}
\end{center}

We enumerates the space fullerenes with a small fundamental domain
under their translation groups. The technique is decomposed into
two steps:
\begin{itemize}
\item The use of Delaney symbols, i.e. flag systems, as the basic
model for storing combinatorial informations and the use of the softwares
{\bf 3dt} and {\bf Systre} by Olaf Delgado for transforming
combinatorial description into topological and metric ones.

\item The design of programs for enumerating space-fullerenes, using
all the tricks in the books of combinatorial enumeration in order to
achive the highest speed possible.
\end{itemize}

\end{abstract}




\end{document}






